\documentclass[11pt]{article}
\usepackage{minted} 	     % for code blocks, syntax highlighting
\usemintedstyle{vs}
\usepackage{xcolor}
\usepackage{fullpage}  	     % make margins smaller
\usepackage{parskip} 	     % disable paragraph indentation
\usepackage{hyperref}
\hypersetup{
    bookmarks=true,         % show bookmarks bar?
    unicode=false,          % non-Latin characters in Acrobat’s bookmarks
    pdftoolbar=true,        % show Acrobat’s toolbar?
    pdfmenubar=true,        % show Acrobat’s menu?
    pdffitwindow=false,     % window fit to page when opened
    pdfstartview={FitH},    % fits the width of the page to the window
    pdfnewwindow=true,      % links in new PDF window
    colorlinks=true,       	 % false: boxed links; true: colored links
    linkcolor=blue,         % color of internal links (change box color with linkbordercolor)
    citecolor=red,     	     % color of links to bibliography
    filecolor=red,          % color of file links
    urlcolor=purple         % color of external links
}
\usepackage[none]{hyphenat} % Disable hyphenation

\newminted{ruby}{bgcolor=lightgray!40}
\newcommand{\snippet}[1]{\colorbox{lightgray!40}{#1}}
\newcommand{\chead}[1]{\fbox{\textit{#1}}}
\newcommand{\sectionkeyword}[1]{ \vspace{3mm} \centerline{\colorbox{black}{\color{white}{\textbf{{\large #1}}}}} \vspace{3mm} }

\author{Matthew Nielsen}
\title{Ruby notes}
\date{\today}
\begin{document}
\maketitle
\begin{center}
License: \href{https://creativecommons.org/licenses/by/4.0/}{Creative Commons}
\end{center}
\tableofcontents

\section{Ruby Basiheadercs}
\subsection{Introduction}
Interactive shell: \snippet{irb} \\
File extension: \snippet{.rb}

\subsection{Comments}
Commenting is done via \snippet{\#} symbol.
Multi line comments: \\
\begin{rubycode}
=begin
Everything here is a comment.
The =begin and =end here are not recognized.
It has to be at the beginning of the line to be recognized as a keyword.
=end
puts "This is code"
\end{rubycode}


\subsection{Conditional Statements}

\sectionkeyword{if}
\begin{rubycode}
if 1==1
  puts "true"
elsif 1==2
  puts "1==2"
else
  puts "1 != 1 and 1 != 2"    
\end{rubycode}

Ternary operator: \\
\begin{rubycode}
result = 1==1 ? "true" : "false"
\end{rubycode}


Its arity = 3, so it doesn't anything beyond true and false.

\sectionkeyword{unless}
Takes two operands. Executes first operand if second operand is false. \\
\begin{rubycode}
world_doesnt_exist = true
puts "Hello world" unless world_doesnt_exist
\end{rubycode}

\sectionkeyword{case}

\begin{rubycode}
case expr0
when expr1, expr2
  # stmt1
when expr3, expr4
  # stmt2
when expr5
  # stmt3    
else
  # stmt4  
end
\end{rubycode}


This has the same function as the following: \\
\begin{rubycode}
if expr0 == expr1 || expr0 == expr1
  # stmt1
elif expr0 == expr3 || expr0 == expr4
  # stmt2
elif expr0 == expr5
  # stmt3
else
  # stmt4
end        
\end{rubycode}

\subsection{Output}

\end{document}